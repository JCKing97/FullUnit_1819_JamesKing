%%%%%%%%%%%%%%%%%%%%%%%%%%%%%%%%%%%%%%%%%
% Journal Article
% LaTeX Template
% Version 1.4 (15/5/16)
%
% This template has been downloaded from:
% http://www.LaTeXTemplates.com
%
% Original author:
% Frits Wenneker (http://www.howtotex.com) with extensive modifications by
% Vel (vel@LaTeXTemplates.com)
%
% License:
% CC BY-NC-SA 3.0 (http://creativecommons.org/licenses/by-nc-sa/3.0/)
%
%%%%%%%%%%%%%%%%%%%%%%%%%%%%%%%%%%%%%%%%%

%----------------------------------------------------------------------------------------
%	PACKAGES AND OTHER DOCUMENT CONFIGURATIONS
%----------------------------------------------------------------------------------------

\documentclass[twoside,twocolumn]{article}

\usepackage{blindtext} % Package to generate dummy text throughout this template 

\usepackage[sc]{mathpazo} % Use the Palatino font
\usepackage[T1]{fontenc} % Use 8-bit encoding that has 256 glyphs
\linespread{1.05} % Line spacing - Palatino needs more space between lines
\usepackage{microtype} % Slightly tweak font spacing for aesthetics

\usepackage[english]{babel} % Language hyphenation and typographical rules

\usepackage[hmarginratio=1:1,top=32mm,columnsep=20pt]{geometry} % Document margins
\usepackage[hang, small,labelfont=bf,up,textfont=it,up]{caption} % Custom captions under/above floats in tables or figures
\usepackage{booktabs} % Horizontal rules in tables

\usepackage{lettrine} % The lettrine is the first enlarged letter at the beginning of the text

\usepackage{enumitem} % Customized lists
\setlist[itemize]{noitemsep} % Make itemize lists more compact

\usepackage{abstract} % Allows abstract customization
\renewcommand{\abstractnamefont}{\normalfont\bfseries} % Set the "Abstract" text to bold
\renewcommand{\abstracttextfont}{\normalfont\small\itshape} % Set the abstract itself to small italic text

\usepackage{titlesec} % Allows customization of titles
\renewcommand\thesection{\Roman{section}} % Roman numerals for the sections
\renewcommand\thesubsection{\roman{subsection}} % roman numerals for subsections
\titleformat{\section}[block]{\large\scshape\centering}{\thesection.}{1em}{} % Change the look of the section titles
\titleformat{\subsection}[block]{\large}{\thesubsection.}{1em}{} % Change the look of the section titles

\usepackage{fancyhdr} % Headers and footers
\pagestyle{fancy} % All pages have headers and footers
\fancyhead{} % Blank out the default header
\fancyfoot{} % Blank out the default footer
\fancyhead[C]{Running title $\bullet$ May 2016 $\bullet$ Vol. XXI, No. 1} % Custom header text
\fancyfoot[RO,LE]{\thepage} % Custom footer text

\usepackage{titling} % Customizing the title section

\usepackage{hyperref} % For hyperlinks in the PDF

%----------------------------------------------------------------------------------------
%	TITLE SECTION
%----------------------------------------------------------------------------------------

\setlength{\droptitle}{-4\baselineskip} % Move the title up

\pretitle{\begin{center}\Huge\bfseries} % Article title formatting
\posttitle{\end{center}} % Article title closing formatting
\title{System design report: The Prolog service and Agents, and the web application and agent's environment} % Article title
\author{%
\textsc{James King} \\% Your name
\normalsize Supervisor: Kostas Stathis \\ % Your supervisor
}
\date{October 2018} % Leave empty to omit a date
\renewcommand{\maketitlehookd}{%
\begin{abstract}
\noindent Prolog services, Flask web application\\
https://users.ece.cmu.edu/~koopman/essays/abstract.html
\end{abstract}
}

%----------------------------------------------------------------------------------------

\begin{document}

%Milestone: Report on the design of the Prolog service and Agents, and the web application and agent’s environment
%Will include the following: an agent definition and the inner workings of an agent’s decision-making component, a definition of the structure of the Prolog service and management of agents (as well as the design of the API), a definition of the agent’s environment and how this will work with the Prolog service, and a design of the web application. This is key to a well-planned architecture for the Prolog service and agent environment. It will also help me research better methods of implementation for the Prolog service.
% Link to goal: This will lay out a concrete way of implementing the Prolog service, web application and link between them.

% Print the title
\maketitle

%----------------------------------------------------------------------------------------
%	ARTICLE CONTENTS
%----------------------------------------------------------------------------------------

\section{Introduction}


%------------------------------------------------

\section{Contents and Knowledge}
Talk about choice between kostas' idea and mine.\\
Talk about proof of concept applications and lessons learnt from them.\\
Talk about learning from Miguel Grinberg and Annie Ogborne.
\subsection{Prolog Service}
Talk about security: Prolog injection attacks, shell injection attacks, sql injection, remove library(http/http_errors.pl) in production to make it harder


\subsection{Agent Design}

\subsection{Flask application}
Talk about security: Prolog injection attacks, shell injection attacks, sql injection, 


\subsection{Environment}

\subsection{Communication}

\subsection{Testing}
Postman and unittest


%------------------------------------------------

\section{Discussion and Conclusion}
Secur


%----------------------------------------------------------------------------------------
%	REFERENCE LIST
%----------------------------------------------------------------------------------------

\bibliography{../refs.bib}{}
\bibliographystyle{plain}

%----------------------------------------------------------------------------------------

\end{document}
