\documentclass{article}
\usepackage[utf8]{inputenc}
\usepackage{color}
\usepackage[margin=1in]{geometry}
\usepackage{url}

\title{Project Plan}
\author{James King}
\date{September 2018}

\begin{document}

\maketitle

\section*{Abstract}

\subsection*{Background}
The Iterated Prisoner's Dilemma uses a mechanism known as direct reciprocity to aid in the evolution of cooperation, this was popularized in Axelrod's tournaments and research on it~\cite{evolution_of_cooperation}. There is a well-tested library that is able to run matches, tournaments and the Moran Process for the Iterated Prisoner's Dilemma~\cite{axelrodproject}. This is not the only game-theory mechanism used to model the evolution of cooperation. Nowak presents 4 alternative methods~\cite{five_rules_coop} including kin selection, group selection, network reciprocity and indirect reciprocity.\\\\
Of these indirect reciprocity stands out as an interesting model for interaction between humans or intelligent agents. Pairs of agents are chosen at random to interact (they may not necessarily interact again), in each pair one agent is the donor and the other is the recipient. A donor chooses whether to defect or cooperate, defecting gives them a higher payoff so is always tempting, however, the model utilizes the idea of reputation (also known as image score~\cite{evol_indirect_image}) to create a condition where an agent may want to cooperate. If the donor cooperates their reputation will be enhanced if they defect it will fall. Some strategies are known as discriminators, these will choose to defect if the recipient has a low reputation or cooperate if they have a good reputation.\\\\
Nowak and Sigmund present the idea of onlookers~\cite{evol_indirect_image} this limits the number of direct observers of interactions. As is noticeable in real life, direct observation of other's interactions is not always possible, so this seems to more closely match real life. In societies with advanced communication however direct observation is not the only method of finding out about interactions, gossip has been put forward as an alternative form~\cite{gossip_alt}. Previously the outcome depended upon the probability of witnessing another's decision, now with gossip, it depends upon how the information has been spread and how agents base their behaviour on gossip. Ralf D. Sommerfeld et. al state that gossip needs to truthfully describe other's behaviour, and when hearing correct gossip an agent should act accordingly or gossip will fail. It is also suggested that gossip can be kept within the boundaries of truth if spreading misinformation leads to a loss of reputation.\\\\
In thinking about how to implement this model, we will consider using the logic programming language Prolog  - because it has the potential to provide transparent decisions - to implement an agent's decision-making component was most appropriate, whilst implementing the environment itself in a procedural language would be more appropriate (Python has been chosen). The issue with this is interfacing between the two languages, there are libraries such as PySwip~\cite{pyswip} that are able to do this but are limited. For instance PySwip struggles to work in multi-threaded environments.

\subsection*{Project Description}
I will explore indirect reciprocity, by both researching and reporting on Nowak and other authors research on the mechanism, drawing on their research and possibly adding on to the ideas of onlookers and gossip to create a model for implementation. I shall be using the Axelrod-Python library to create a proof of concept that I am able to create a web application that allows users to set up and run games and tournaments. This web application will be hosted on Heroku and created using Flask and Python.\\\\
The solution that we are exploring for the interfacing issues between Python and Prolog is to create a web service for the agent's decision-making component. This will have a Prolog server containing ``agent's minds". The minds themselves will represent information on the state of the environment using an efficient version of the events calculus that Kostas has worked on (mvfcec). Percepts can be added to the agent's mind and based on these percepts an agent will be able to decide on an action and act on it by returning an action to the environment. Separating out the head and body as in Stathis \textit{et al.} 2002~\cite{prosocs}, though this solution would take this separation further, having the head based in the Prolog service and the body in the environment.\\\\
The capabilities of the Prolog service are to be demonstrated on the web application, which will run indirect reciprocity tournaments creating the environment that agents reside in and containing the body of the agent. The web application will use the service to input percepts and get actions from.\\\\
These have been the core parts of the project however, I could extend this to the other mechanisms presented by Nowak~\cite{five_rules_coop} if I have time. Axelrod-Python has an implementation for network reciprocity - which would be interesting to implement so I could attempt to create an especially nice frontend for the patterns created in the population~\cite{spatial}. Due to the nature of group selection working on top of direct reciprocity~\cite{multilevel_nowak} it would be possible to implement it on top of the Axelrod-Python library. Leaving kin selection to implement using the Prolog service.

\subsection*{Goals}
\begin{itemize}
\item To create a Prolog service that is able to support the decisions of a game-theoretic agent to interact with any other system that is capable of creating an environment for that agent to work in, and contacting the service.
\item To explore in a game-theoretic context a mechanism to aid in the evolution of cooperation known as indirect reciprocity.
\item To create a web application that allows users to set up and run game-theory tournaments of both direct and indirect reciprocity.
\item To use the web application to demonstrate an environment that combines the Prolog service and a set of game-theory strategies and evaluate the results.
\end{itemize}

\subsection*{Motivation}
The development of the Prolog service will provide a solution to problems interfacing between Prolog and other languages, meaning it may be a useful model for intelligent agents and multi-agent systems research. It will develop my logic programming skills, this is key to my interest in intelligent agents, aiding my aim to secure a job in the AI sector.\\\\
Understanding and modelling interaction between agents is important to the development of AI systems, indirect and direct reciprocity are two aids to this, thus understanding and implementing them will aid me in my aim to secure a job in AI. Not only this but due to my research I have developed a keen interest in game theory and studying interactions between biological agents, especially human interaction.\\\\
Web development is an area of interest for me as well and is a sector in which I'd be keen to go into. Developing a web application will help me develop my skills as a developer in the full stack, whilst the Prolog service will augment my skills in developing web services.

\section*{Timeline}

I'm approaching this project in an agile way, using 2-week sprints to complete tasks, as such the milestones here shall be assigned to a sprint week or multiple sprint weeks if they are more likely to take longer. This does not mean that I will not be working on a task beforehand if possible, but that I plan to finish that task by the end of the later sprint. In the winter period, I will take my time to consolidate and finish current work, whilst planning the next terms work. Here are the timings for each sprint:
\begin{center}
	\begin{tabular}{c}
	Sprint 1: 1-12 October \\
	Sprint 2: 15-26 October\\
	Sprint 3: 29 October - 9 November\\
	Sprint 4: 12-23 November\\
	Sprint 5: 26 November - 7 December\\
	\end{tabular}
	\begin{tabular}{c}
	Sprint 7: 14-15 January\\
	Sprint 8: 28 January - 8 February\\
	Sprint 9: 11-22 February\\
	Sprint 10: 25 February - 8 March\\
	Sprint 11: 11-22 March\\
	\end{tabular}
\end{center}

\subsection*{First Term}

\noindent\textbf{Milestone: A proof of concept program for running a Prolog web service}\\
I will be using tutorials online to learn how to create a Prolog web service, and from this learning, I will attempt to create my own service. This service should resemble a very early version of the Prolog service I wish to provide. This will prove to myself that I can create a Prolog web service and give me confidence, but also reduce the risk of the project failing if I cannot create one.\\
Date: Sprint 1\\

\noindent\textbf{Milestone: A proof of concept web application to run direct reciprocity games and tournaments and provide statistics on them}\\
The environment for the agents will be on a web application, this proof of concept will show that I am capable of creating a web application that can run similar games without the need to program the environment inside it yet. I will be using the Axelrod-Python library to run the games for a convincing mock-up.\\
Date: Sprint 1\\

\noindent\textbf{Milestone: Report on the evolution of cooperation in relation to game-theory and different game-theoretic mechanisms to aid it}\\
Report on past work in this area such as Axelrod's tournaments~\cite{evolution_of_cooperation} and Nowak's five rules for the evolution of cooperation~\cite{five_rules_coop}. This will help motivate my project as to how I will implement the web application.\\
Date: Sprint 1\\

\noindent\textbf{Milestone: Report on indirect reciprocity, strategies for agents and the development of a concrete model to implement}\\
Report on past work on indirect reciprocity including the strategies agents have employed playing this game. From this collated work I shall define a model which I am to implement and a number of key strategies to implement as well.\\
Date: Sprint 2\\

\noindent\textbf{Milestone: Report on the design of the Prolog service and Agents, and the web application and agent's environment}\\
Including an agent definition and the inner workings of an agent's decision-making component, a definition of the structure of the Prolog service and management of agents (as well as the design of the API), a definition of the agent's environment and how this will work with the Prolog service, and a design of the web application. This is key to a well-planned architecture for the Prolog service and agent environment. It will also help me research better methods of implementation for the Prolog service.\\
Date: Sprint 2\\

\noindent\textbf{Milestone: Design and implementation of a GUI for indirect reciprocity tournaments}\\
This will follow on from my proof of concept web application, to which I will add a GUI to set up, run and provide analysis on indirect reciprocity tournaments. This will help motivate the creation of the backend for running the environment and the Prolog service in turn.\\
Date: Sprints 2 and 3\\

\noindent\textbf{Milestone: Design and implementation of the structure of the Prolog service}\\
Taking inspiration from my proof on concept Prolog service and previous architectures for intelligent agents and multi-agent systems~\cite{prosocs} I will define the structure of the Prolog service programmatically.\\
Date: Sprints 3 and 4\\

\noindent\textbf{Milestone: Design, implementation and testing of the management of agents and agent templates in the Prolog service}\\
Once the structure of the Prolog service has been implemented I can then develop the components that will manage agents in the system. The system will have to deal with the management of multiple agents all at one time so this must be carefully implemented and thoroughly tested (I will need to learn how to test a Prolog system for this). For agents to work correctly in the system they will have to follow a regular pattern (naming and functionality). This pattern I will define in a template in the Prolog service.\\
Date: Sprints 3 and 4\\

\noindent\textbf{Milestone: Design, implementation and testing of the environment for a simple indirect reciprocity tournament}\\
The environment is a key aspect of the overall product, to run the indirect reciprocity tournaments. It will run the cycles of the game and record information about the game. This will be written in the web application and needs to be thoroughly tested.\\
Date: Sprints 3 and 4\\

\noindent\textbf{Milestone: Design, implementation and testing of defector, cooperator and discriminator strategies for a simple indirect reciprocity tournament}\\
We need strategies in the Prolog service to play indirect reciprocity tournaments. The simplest of which are defector and cooperator strategies, whilst discriminator strategies make indirect reciprocity tournaments varied, which is why I shall start by implementing them using the agent template in the Prolog service.\\
Date: Sprint 4\\

\noindent\textbf{Milestone: Linking together the Prolog service and web application environment}\\
After the implementation of a subset of indirect reciprocity functionality - without onlookers and gossip - I should now be ready to link together the tested web application environment and Prolog service.\\
Date: Sprints 4 and 5\\

\noindent\textbf{Milestone: Preparing the interim report and presentation}\\
With the programs working together and finished first term reports I shall prepare the reports and presentation for the interim review.\\
Date: Sprint 5\\

\subsection*{Second Term}

\noindent\textbf{Milestone: Design, implementation and testing of the onlooker and gossip features in the web application environment}\\
The model that I'm exploring becomes more interesting with onlookers and gossip features, with the simple version of the web application and Prolog service linked together I can now move onto developing the onlooker and gossip features. These features need support from the web application environment.\\
Date: Sprints 7 and 8\\

\noindent\textbf{Milestone: Design, implementation and testing of the onlooker and gossip features in the Prolog service}\\
While developing the onlooker and gossip features in the web application environment, in parallel I can develop the Prolog service features to support these features. All the time ensuring that both the environment and the Prolog service can work together.\\
Date: Sprints 7 and 8\\

\noindent\textbf{Milestone: Design, implementation and testing of a development upon current strategies to play the gossip and onlookers meta-game}\\
The onlookers and gossip features need agents to support that functionality, so they can spread gossip, listen to gossip and have appropriate reactions for this gossip.\\
Date: Sprints 7 and 8\\

\noindent\textbf{Milestone: Relinking the Prolog service and web application environment with the onlooker and gossip meta-game implemented}\\
With the onlookers and gossip features fully support by agents, the Prolog service and the environment, I can move onto linking them together. Ensuring there aren't issues communicating and that the applications work together correctly.\\
Date: Sprint 8\\

\noindent\textbf{Milestone: Implementation and testing of further strategies for an indirect reciprocity tournament}\\
So far we have 3 strategies, but there are other strategies available for these tournaments that don't just base their actions on the recipient's reputation but also their own reputation or even both. These will again use the agent template and be implemented in the Prolog service.\\
Date: Sprints 8 and 9\\

\noindent\textbf{Milestone: Enhanced analysis post-tournament for indirect reciprocity}\\
Part of the reason for using Prolog for the agent' decision-making component is that we are hopefully able to query why an agent has made the decision it has. In the analysis post-tournament using an Ajax request, users should be able to query the decisions. Furthermore, analysis of games should be enhanced using graphs and possibly animation to display a timeline of actions. This will enhance the capabilities of the system in explaining what interactions occurred and why they occurred as they did.\\
Date: Sprints 8 and 9\\

\noindent\textbf{Milestone: Enhanced GUI features}\\
I will have focused a lot on developing the features of the application. With these features now implemented I can work on giving a facelift to the web application using bootstrap, React.js and CSS. This will enhance the web applications abilities as an education system and also be a good preparation for the demonstration.\\
Date: Sprints 8 and 9\\

\noindent\textbf{Milestone: The report should describe the theory behind indirect reciprocity and it's strategies in relation to game-theory}\\
Indirect reciprocity is a game-theoretic modelling tool for interactions, it is key to understanding the project to understand this model and the strategies used in it.\\
Date: Sprints 9 and 10\\

\noindent\textbf{Milestone: The report should provide an analysis and evaluation of strategies in indirect reciprocity tournaments}\\
While implementing the project I will have gained in-depth knowledge of the strategies in indirect reciprocity and should be able to analyse them and evaluate their effectiveness in indirect reciprocity tournaments. This should use examples of tournaments run by my system.\\
Date: Sprints 9 and 10\\

\noindent\textbf{Milestone: The report should describe the onlookers and gossip aspects of indirect reciprocity}\\
These extensions on indirect reciprocity are an area of interest for understanding how the spread of information affects decisions made by agents. Understanding these extensions is key to understanding a lot of the motivation behind my project.\\
Date: Sprints 9 and 10\\

\noindent\textbf{Milestone: The report should describe the link between indirect reciprocity theory and real life biological and intelligent agent interactions}\\
Indirect reciprocity is a modelling tool, but how is this modelling tool useful to understand real interactions between agents? In this section, I wish to describe the link between the theory and the real world, its positives and also it's limitations.\\
Date: Sprints 9 and 10\\

\noindent\textbf{Milestone: The report should contain a design of the web application and environment, Prolog service and agents, and the connection between them}\\
Program design is key to developing understandable, working applications. The design of the system as a whole and also individual programs have been motivated by past research on multi-agent systems and the difficulties I will come across in creating one.\\
Date: Sprints 9 and 10\\

\noindent\textbf{Milestone: The report should contain a discussion of the software engineering techniques, tools and processes used and issues encountered}\\
The use of software engineering techniques, tools and processes is key to writing good code that works. In my project, I will be using a variety of them in the hope of writing good code that works. Here I will review my use of them, both successes and failures.\\
Date: Sprints 9 and 10\\

\noindent\textbf{Milestone: The report should contain any interesting programming techniques employed to develop the final prototype}\\
To create the system I will need to employ techniques used in web development and logic programming among other fields. With such a wide variety of techniques available this will be interesting to review and report on.\\
Date: Sprints 9 and 10\\

\noindent\textbf{Milestone: Draft report}\\
A draft report will be a useful point to review what I have written so far. Allowing me to check my progress and develop on what I have written.\\
Date: Sprint 9\\

\noindent\textbf{Milestone: Development of content for the web application to make it into an educational system for the evolution of cooperation}\\
The website should have content that enhances the user's experience, explaining the motivation behind concepts and features of the application.\\
Date: Sprint 10\\

\noindent\textbf{Sprints 10 and 11}\\
These two sprints will be dedicated to finishing unfinished parts of the end products, the report as well as preparing for the demonstration. This period should act as a buffer if I fall behind, or a period to perfect or extend what I have produced if not.

\section*{Risk Assessment}

\subsection*{Risk: Using SWI-Prolog to run a web service}
This is new to both me and my supervisor. It has been done before and there are resources online, but it is still new territory to me. Writing web services is also new territory to me as well. This could delay the delivery of the Prolog service.\\
Mitigation: If I fail to be able to deliver this on time I shall switch my plan to one of the following two options: write a service in Python to replace the Prolog service or forget the service and code the program to run indirect reciprocity directly in the web application.\\
Likelihood: Possible\\
Importance: Moderate

\subsection*{Risk: Failure to understand and use mvfcec}
Mvfcec or the multi-value fluent cached event calculus is a version of the event calculus where querying is faster. I have knowledge of the event calculus itself, but this is a more complicated form of it. If I fail to understand and use it, I cannot write a report on it and either the Prolog service will be far slower or I will fail to deliver it\\
Mitigation: I shall attempt to avoid this by working with Kostas to understand it. However, if I still fail to grasp mvfcec, I shall either switch to using the simplified events calculus or use one of the two mitigations from the risk: Using SWI-Prolog to run a web service.
Likelihood: Possible\\
Importance: Moderate

\subsection*{Risk: My inexperience of writing large Prolog base applications}
I have used Prolog in the AI course in my second year, however, these applications were nowhere near as large or complex as the Prolog service will have to be. I am doing it this way to push myself, but recognise this could cause a failure to deliver the service.\\
Mitigation: switch from Prolog to Python for writing the service or simply include the agent decision-making component in the web application.\\
Likelihood: Possible\\
Importance: Moderate

\subsection*{Risk: My inexperience in writing large web applications}
In my first year, we used HTML, CSS, Javascript etc. to learn about writing web applications. However, writing a large back and front end of a web application is new to me. Meaning there will be obstacles that I have not yet anticipated. \\
Mitigation: I have taken the Miguel Grinberg Flask Mega-Tutorial, to further practice my skills in writing web applications. I feel that my practice in this has allowed me to give a low likelihood that this will affect my project, and due to the resources available for Flask the importance of any obstacles to my project will be minor.\\
Likelihood: Unlikely\\
Importance: Minor

\subsection*{Risk: The scale of the project}
With so many new technologies and techniques to learn, programs to implement and so much theory to understand this project I believe to be quite large. If this project is too large it could delay the delivery of multiple areas of the project.\\
Mitigation: I have already started on much of the learning for the project, which should help reduce the time for learning. If there is too much to implement for the project programs I will have to carefully scale down the project parts at a time, this may include: reducing the features of the indirect reciprocity tournaments, removing the Prolog service and replacing it with functionality inside the web application or a reduction to the 3 key strategies (defector, cooperator and discriminator). To manage this possible scaling down or up of the program I have gone for an agile approach using scrum techniques.\\
Likelihood: Likely\\
Importance: Minor (due to the many mitigations I can employ, the impact should be minimal)

\bibliography{refs.bib}{}
\bibliographystyle{plain}

\subsection*{Of what value are these references to the project?}
\cite{evolution_of_cooperation} - This paper and Axelrod's tournaments helped popularize the iterated prisoner's dilemma, being a good introduction to game theory that has a programmatic approach for computer science.\\
~\cite{axelrodproject} - This project is a relatively new implementation of the iterated prisoner's dilemma, pushing me towards focusing on other mechanisms, whilst helping me write a proof of concept web application.\\
~\cite{five_rules_coop} - This paper introduced me to the other mechanisms that aid in modelling the evolution of cooperation.\\
~\cite{spatial} - This paper extended my research on network reciprocity and pushed me towards an interest in indirect reciprocity.\\
~\cite{evol_indirect_image} - This paper is key to understanding and reasoning about indirect reciprocity models and is central to past research on this mechanism.\\
~\cite{gossip_alt} - This paper gives a good explanation of the concept of gossip in the spread of information as well as how gossip should be used.\\
~\cite{prosocs} - This paper is on an architecture for the design of a multi-agent system and has helped motivate my ideas on how to design and implement the Prolog service.\\
~\cite{pyswip} - This library among others attempts to implement an interface from Python to Prolog. It has issues and this has motivated me to find a solution to these issues.\\
~\cite{multilevel_nowak} - This paper is a good description of a model for group selection and has given me an idea for the group level selection extension to my project.\\

\end{document}